\documentclass[1]{mathtoolkit}

\begin{document}

\maketitle

\begin{p}
  \item
    \begin{subp}
      \item
        \begin{gather}
          dim(A) = rank(A) + null(A) \\
          n = m + null(A) \\
          null(A) = n - m
        \end{gather}
      \item
        \begin{gather}
          null(A) = dim(ker(A))
        \end{gather}
        Then, $ker(A)$ can have a basis $B$ s.t. $Span(B) = ker(A)$.
        i.e.
        \begin{gather}
          \forall \vec{v} \in ker(A),
          \exists a_1, ..., a_{n-m} \in \field{F}_2, \\
          \vec{v} = a_1 \vec{b}_1 + \dots + a_{n-m} \vec{b}_{n-m} (b_i \in B)
        \end{gather}
        \therefore The answer is $2^{n-m}$.
      \item
        \begin{gather}
          \forall \vec{x}
          \text{s.t.}
          \begin{cases}
            A \vec{x} = \vec{b} \\
            A \vec{x_0} = \vec{b}
          \end{cases} \\
          \therefore A (\vec{x} - \vec{x}_0) = 0 \\
          \vec{x} - \vec{x}_0 \in ker(A)
        \end{gather}
        Then, choosing each element of $\vec{x}$ carefully (1 or 0),
        $\vec{x} - \vec{x}_0$ can be any element of $\field{F}^n_2$.
        \begin{gather}
          \therefore \left\{ \vec{x} - \vec{x}_0 | A \vec{x} = b \right\} = ker(A)
        \end{gather}
        \therefore $\vec{x} - \vec{x}_0$ has $2^{n-m}$ solutions. \\
        \therefore $\vec{x}$ has $2^{n-m}$ solutions. \\
    \end{subp}

  \item
    \begin{subp}
      \item
        \begin{gather}
          f(c \vec{v} + (-c) \vec{v}) \ge \min \left\{ f(\vec{v}), f(\vec{v}) \right\} \\
          \therefore f(\vec{0}_V) \ge f(\vec{v})
        \end{gather}
      \item
        Because every element $\vec{v}_t \in V_t$ is in V by definition.
        \begin{gather}
          V_t \subseteq V
        \end{gather}
    \end{subp}

  \item
    \begin{align}
      p(x) & = x^2 + bx +c \\
           & = (x - r_1) (x - r_2) \\
           & = x^2 - (r_1 + r_2) x + r_1 r_2 \\
    \end{align}
    \begin{gather}
      \therefore b = - r_1 - r_2, c = r_1 r_2 \\
    \end{gather}

  \item
    \begin{align}
      \mu(P, Q) & = degree(PQ) \\
                & = degree(QP) \\
                & = \mu(Q, P)
    \end{align}
    \begin{align}
      \mu(0, 0) & = degree(0) \\
                & = 0
    \end{align}
    \begin{align}
      \forall P \ne 0, \\
      \mu(P, P) & = degree(P^2) \\
                & = 2 degree(P) \\
                & > 0
    \end{align}
    \begin{align}
      \mu(P + Q, R) & = degree((P + Q) R) \\
                    & = \max \left\{ degree(P), degree(Q) \right\} + degree(R)
    \end{align}
    \begin{align}
      \mu(P, R) + \mu(Q, R) & = \max \left\{ degree(P) + degree(R), degree(Q) + degree(R) \right\} \\
                            & = \max \left\{ degree(P), degree(Q) \right\} + degree(R)
    \end{align}
    \begin{gather}
      \therefore \mu(P + Q, R) = \mu(P, R) + \mu(P, Q)
    \end{gather}
    \begin{align}
      c \in \field{R}, \\
      \mu(cP, R) & = degree(cPR) \\
                 & = degree(PR) \\
                 & \ne c \cdot degree(P, R)
    \end{align}
    \therefore $\mu(\cdot, R)$ is not a LT. \\
    \therefore $\mu$ is not a IP.

  \item
    \begin{gather}
      \alpha \beta \vec{x} = \lambda \vec{x} \\
      \beta \alpha \beta \vec{x} = \beta (\lambda \vec{x}) \\
      \beta \alpha (\beta \vec{x}) = \lambda (\beta \vec{x}) \\
    \end{gather}
    \therefore $\lambda$ is an eigenvalue of $\beta \alpha$.

  \item
    \begin{subp}
      \item
        \begin{gather}
          \varphi(\vec{v}) = \lambda \vec{v} \\
          \varphi(\vec{v}) = \lambda \varphi(\vec{v}) \\
          (\lambda - 1) \varphi(\vec{v}) = 0 \\
          \lambda = 1_\field{F} \vee \varphi(\vec{v}) = 0 \\
          \lambda = 1_\field{F} \vee \varphi(\vec{v}) = 0_\field{F} \vec{v} \\
          \therefore \lambda \in \left\{ 0_\field{F}, 1_\field{F} \right\}
        \end{gather}

      \item
        Let $\forall \vec{v}, \varphi(\vec{v}) = \vec{v}_0$.
        ($\vec{v}_0$ is fixed.) \\
        Then assume $\varphi = \varphi^*$. \\
        If $\vec{v} \ne \vec{w} \in V$,
        \begin{gather}
          \ip{\vec{v}_0}{\vec{w}} = \ip{\vec{v}}{\varphi^*(\vec{w})} \\
          \ip{\vec{v}_0}{\vec{w}} = \ip{\vec{v}}{\vec{v}_0} \\
          \ip{\vec{w}}{\vec{v}_0} = \ip{\vec{v}}{\vec{v}_0} \\
          \vec{v} = \vec{w}
        \end{gather}
        This is contradition. \\
        \therefore not always $\varphi = \varphi^*$.
    \end{subp}

  \item
    \begin{subp}
      \item
        \begin{gather}
          \ip{\varphi(\vec{v})}{\vec{w}} = \ip{\vec{v}}{\varphi^*(\vec{w})} \\
          \ip{\varphi^*(\vec{w})}{\vec{v}} = \ip{\vec{w}}{\varphi(\vec{w})} \\
          \therefore (\varphi^*)^* = \varphi
        \end{gather}
    \end{subp}
\end{p}

\end{document}
