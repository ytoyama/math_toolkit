\documentclass{mathtoolkit}
\nth{1}

\newcommand\row{\mathrm{row}}
\newcommand\column{\mathrm{column}}

\begin{document}

\maketitle

\begin{p}
  \item
    \begin{subp}
      \item
        \begin{gather*}
          \dim{A} = \rank{A} + \mnull{A} \\
          n = m + \mnull{A} \\
          \mnull{A} = n - m
        \end{gather*}
      \item
        \begin{gather*}
          \mnull{A} = \dim{\ker{A}}
        \end{gather*}
        Then, $\ker{A}$ can have a basis $B$.
        \begin{gather*}
          \text{i.e.} \forall \vec{v} \in \ker{A},
          \exists a_1, ..., a_{n-m} \in \field{F}_2,
          \vec{v} = a_1 \vec{b}_1 + \dots + a_{n-m} \vec{b}_{n-m}\ (\vec{b}_i \in B)
        \end{gather*}
        \therefore The answer is $2^{n-m}$.
      \item
        \begin{gather*}
          \forall \vec{x} \st A \vec{x} = \vec{b},
          A (\vec{x} - \vec{x}_0) = 0 \because A \vec{x}_0 = \vec{b} \\
          \therefore \vec{x} - \vec{x}_0 \in \ker{A}
        \end{gather*}
        Then, choosing each element of $\vec{x}$ carefully (1 or 0),
        $\vec{x} - \vec{x}_0$ can be any element of $\field{F}^n_2$.
        \begin{gather*}
          \therefore \left\{ \vec{x} - \vec{x}_0 | A \vec{x} = \vec{b} \right\} = \ker{A}
        \end{gather*}
        \therefore $\vec{x} - \vec{x}_0$ has $2^{n-m}$ solutions. \\
        \therefore $\vec{x}$ has $2^{n-m}$ solutions. \\
    \end{subp}

  \item
    \begin{subp}
      \item
        \begin{align*}
          \forall \vec{v},
          f(c \vec{v} + (-c) \vec{v}) & \ge \min \left\{ f(\vec{v}), f(\vec{v}) \right\} \\
          f(\vec{0}) & \ge f(\vec{v})
        \end{align*}
      \item
        \begin{gather*}
          V_t \subseteq V \text{by definition} \\
          \vec{0} \in V_t \because \text{(a)} \\
          \forall t \in [0, f(\vec{0})], \forall \vec{v}, \vec{w} \in V_t, \\
          \begin{aligned}
            f(\vec{v} + \vec{w}) & \ge \min \left\{ f(\vec{v}), f(\vec{w}) \right\} \\
                                 & \ge t
          \end{aligned} \\
          \therefore \vec{v} + \vec{w} \in V_t \\
          \begin{aligned}
            \text{and} \forall c \in \field{F},
            f(c\vec{v}) & \ge \min \left\{ f(\vec{v}), f(\vec{0}) \right\} \\
                        & \ge f(\vec{v}) \\
                        & \ge t
          \end{aligned} \\
          \therefore c \cdot \vec{v} \in V_t \\
          f(- \vec{v}) \ge t \text{as above} \\
          \therefore - \vec{v} \in V_t
        \end{gather*}
        \therefore $V_t$ is a subspace of $V$.
    \end{subp}

  \item
    \begin{gather*}
      \begin{aligned}
        p(x) & = x^2 + bx +c \\
             & = (x - r_1) (x - r_2) \\
             & = x^2 - (r_1 + r_2) x + r_1 r_2
      \end{aligned} \\
      \therefore b = - r_1 - r_2, c = r_1 r_2 \\
      \therefore \varphi_1(x_1, x_2, x_3) = x_1 - (r_1 + r_2) x_2 + r_1 r_2 x_3 \\
      \forall f \in \field{R}^\field{N} \text{of an eigenvector of $\varphi_{left}$ with an eigenvalue $\lambda$}, \\
      \begin{aligned}
        (\varphi_2(f))(n)
        & = \varphi_1(f(n), f(n+1), f(n+2)) \\
        & = f(n) - (r_1 + r_2) f(n+1) + r_1 r_2 f(n+2) \\
        & = f(n) - (r_1 + r_2) (\varphi_{left}(f))(n) + r_1 r_2 (\varphi_{left}^2(f))(n) \\
        & = f(n) - (r_1 + r_2) (\lambda f)(n) + r_1 r_2 (\lambda^2 f)(n) \\
        \varphi_2(f)
        & = f - (r_1 + r_2) \lambda f + r_1 r_2 \lambda^2 f \\
        & = (1 - \lambda r_1 - \lambda r_2 + \lambda^2 r_1 r_2) f \\
        & = (1 - \lambda r_1) (1 - \lambda r_2) f
      \end{aligned}
    \end{gather*}
    Let $f_1, f_2 \in \field{R}^\field{N}$ be 2 eigenvectors of $\varphi_{left}$
    with eignevalues $r_1, r_2 \st \forall i \in \{1, 2\}, f_i(n) = r_i^n$.
    \begin{gather*}
      \varphi_2(f_1) = \varphi_2(f_2) = 0 \\
      \left\{ f_1, f_2 \right\} \subseteq \ker{\varphi_2} \\
      \dim{\ker{\varphi_2}} \ge 2
    \end{gather*}

  \item
    \begin{gather*}
      \begin{aligned}
        \mu(P, Q) & = \deg{PQ} \\
                  & = \deg{QP} \\
                  & = \mu(Q, P)
      \end{aligned} \\
      \begin{aligned}
        \mu(0, 0) & = \deg{0} \\
                  & = 0
      \end{aligned} \\
      \begin{aligned}
        \forall P \ne 0, \mu(P, P) & = \deg{P^2} \\
                                   & = 2 \deg{P} \\
                                   & > 0
      \end{aligned} \\
      \begin{aligned}
        \mu(P + Q, R) & = \deg{(P + Q} R) \\
                      & = \max \left\{ \deg{P}, \deg{Q} \right\} + \deg{R}
      \end{aligned} \\
      \begin{aligned}
        \mu(P, R) + \mu(Q, R) & = \max \left\{ \deg{P} + \deg{R}, \deg{Q} + \deg{R} \right\} \\
                              & = \max \left\{ \deg{P}, \deg{Q} \right\} + \deg{R}
      \end{aligned} \\
      \therefore \mu(P + Q, R) = \mu(P, R) + \mu(P, Q) \\
      \begin{aligned}
        c \in \field{R}, \mu(cP, R) & = \deg{cPR} \\
                                    & = \deg{PR} \\
                                    & \ne c \cdot \deg{P, R}
      \end{aligned} \\
    \end{gather*}
    \therefore $\mu(\cdot, R)$ is not a LT. \\
    \therefore $\mu$ is not a IP.

  \item
    \begin{gather*}
      \alpha \beta \vec{x} = \lambda \vec{x} \\
      \beta \alpha \beta \vec{x} = \beta (\lambda \vec{x}) \\
      \beta \alpha (\beta \vec{x}) = \lambda (\beta \vec{x}) \\
    \end{gather*}
    \therefore $\lambda$ is an eigenvalue of $\beta \alpha$.

  \item
    \begin{subp}
      \item
        \begin{gather*}
          \varphi(\vec{v}) = \lambda \vec{v} \\
          \varphi(\vec{v}) = \lambda \varphi(\vec{v}) \\
          (\lambda - 1) \varphi(\vec{v}) = 0 \\
          \lambda = 1_\field{F} \vee \varphi(\vec{v}) = 0 \\
          \lambda = 1_\field{F} \vee \varphi(\vec{v}) = 0_\field{F} \vec{v} \\
          \therefore \lambda \in \left\{ 0_\field{F}, 1_\field{F} \right\}
        \end{gather*}

      \item
        Let $\varphi$ is a projection on $V$ over $\field{F} = \field{R} \text{or} \field{C}$. \\
        $\varphi$ is p.s.d. \because (a) \\
        \begin{gather*}
          \exists \alpha : V \rightarrow V \st \varphi = \alpha^* \alpha \\
          \forall \vec{v}, \vec{w} \in \field{F}^n, \\
          \begin{aligned}
            \ip{\varphi \vec{v}}{\vec{w}} & = \ip{\alpha^*\alpha\vec{v}}{\vec{w}} \\
                                          & = \ip{\vec{v}}{\alpha^*\alpha\vec{w}} \\
                                          & = \ip{\vec{v}}{\varphi^*\vec{w}}
          \end{aligned} \\
          \therefore \varphi = \varphi^*
        \end{gather*}
    \end{subp}

  \item
    \begin{subp}
      \item
        \begin{gather*}
          \ip{\varphi(\vec{v})}{\vec{w}} = \ip{\vec{v}}{\varphi^*(\vec{w})} \\
          \ip{\varphi^*(\vec{w})}{\vec{v}} = \ip{\vec{w}}{\varphi(\vec{v})} \\
          \therefore (\varphi^*)^* = \varphi
        \end{gather*}

      \item
        \begin{gather*}
          \begin{aligned}
            \forall \vec{v} \in \ker{\varphi},
            \varphi(\vec{v}) & = 0 \\
            \ip{\varphi(\vec{v})}{\vec{w}} & = 0 \\
            \ip{\vec{v}}{\varphi^*(\vec{w})} & = 0 \\
            \vec{v} & \in (\im{\varphi^*})^\perp
          \end{aligned} \\
          \begin{aligned}
            \forall \vec{v} \in (\im{\varphi^*})^\perp, \forall \vec{w} \in W,
            \ip{\vec{v}}{\varphi^*(\vec{w})} & = 0 \\
            \ip{\varphi(\vec{v})}{\vec{w}} & = 0 \\
            \varphi(\vec{v}) & = 0 \\
            \vec{v} & \in \ker{\varphi}
          \end{aligned} \\
          \therefore \ker{\varphi} = (\im{\varphi^*})^\perp
        \end{gather*}

      \item
        \begin{gather*}
          \begin{aligned}
            \forall \vec{w} \in \im{\varphi},
            \exists \vec{v} \in V \st \vec{w} = \varphi(\vec{v}),
            \forall \vec{w}' \in \ker{\varphi^*},
            \ip{\vec{w}}{\vec{w}'}
            & = \ip{\varphi(\vec{v})}{\vec{w}'} \\
            & = \ip{\vec{v}}{\varphi^*(\vec{w}')} \\
            & = 0 \\
            \vec{w} & \in (\ker{\varphi^*})^\perp
          \end{aligned} \\
          \begin{aligned}
            % \forall \vec{w} \in (\ker{\varphi^*})^\perp,
            \forall \vec{v} \in V,
            \forall \vec{w} \in \ker{\varphi^*},
            \varphi^*(\vec{w}) & = 0 \\
            \ip{\vec{v}}{\varphi^*(\vec{w})} & = 0 \\
          \end{aligned} \\
          \therefore \im{\varphi} = (\ker{\varphi^*})^\perp
        \end{gather*}

      \item
        \begin{align*}
          \rank{\varphi}
          & = \dim{\im{\varphi}} \\
          & = \dim{(\ker{\varphi^*}}^\perp) \\
          & = \dim{W} - \dim{\ker{\varphi^*}} \\
          & = \dim{im(\varphi^*}) \\
          & = \rank{\varphi^*}
        \end{align*}

      \item
        \begin{gather*}
          \begin{aligned}
            \rank{A} & = \rank{A^*} \because \text{(d)} \\
            \rank[\row]{A} & = \rank[\row]{A^*} \\
            \rank[\row]{A} & = \rank[\column]{\conj{A}}
          \end{aligned} \\
          \begin{aligned}
            \forall n \in \field{N}, \forall \vec{v}, \vec{w} \in \field{C}^n
            \st \ip{\vec{v}}{\vec{w}} & = 0, \\
            \conj{\ip{\vec{v}}{\vec{w}}} & = 0 \\
            \conj{\vec{v}^T \conj{\vec{w}}} & = 0 \\
            \conj{\vec{v}}^T \conj{\conj{\vec{w}}} & = 0 \\
            \ip{\conj{\vec{v}}}{\conj{\vec{w}}} & = 0
          \end{aligned} \\
          \therefore \rank[\row]{A} = \rank[\column]{A}
        \end{gather*}
    \end{subp}
\end{p}

\begin{thebibliography}{9}
\bibitem{mathtoolkit}
    Madhur Tulsiani,
    Mathematical Toolkit - Autumn 2016, \\
    http://ttic.uchicago.edu/~madhurt/courses/toolkit2016/index.html
\bibitem{howtoproove}
    Eugenia Chengi,
    How to write proofs: a quick guide, \\
    http://cheng.staff.shef.ac.uk/proofguide/proofguide.pdf
\bibitem{discussion} Discussion with Tomoki Tsujimura
\end{thebibliography}

\end{document}
